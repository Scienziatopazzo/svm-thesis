\documentclass[12pt]{report}

\usepackage[a4paper]{geometry}
\usepackage{amssymb,amsmath,amsthm}
\usepackage{graphicx}
\usepackage{url}
\usepackage{hyperref}
\usepackage{epsfig}
\usepackage[english]{babel}
\usepackage{tesi}
\usepackage[nottoc]{tocbibind}

\usepackage[utf8]{inputenc}

\newtheorem{myteor}{Theorem}[section]
\newenvironment{teor}{\begin{myteor}\sl}{\end{myteor}}

\begin{document}
\title{SV-based regression techniques for survival analysis: a case study on veterinary data}
\author{Elvis NAVA}
\dept{Corso di Laurea in Informatica} 
\anno{2017-2018}
\matricola{870234}
\relatore{Dario MALCHIODI}
\correlatore{Anna Maria ZANABONI}
%
%        \submitdate{month year in which submitted to GPO}
%		- date LaTeX'd if omitted
%	\copyrightyear{year degree conferred (next year if submitted in Dec.)}
%		- year LaTeX'd (or next year, in December) if omitted
%	\copyrighttrue or \copyrightfalse
%		- produce or don't produce a copyright page (false by default)
%	\figurespagetrue or \figurespagefalse
%		- produce or don't produce a List of Figures page
%		  (false by default)
%	\tablespagetrue or \tablespagefalse
%		- produce or don't produce a List of Tables page
%		  (false by default)
% 
%
\beforepreface
\topskip0pt
\vspace*{\fill}
{\hfill \Large {\sl dedicated to \dots}}
\vspace*{\fill}
% 
%			PREFACE
%
\prefacesection{Preface}
This is a preface.
%
%			ACKNOWLEDGMENTS
%
\prefacesection{Acknowledgments}

\afterpreface


\chapter{Introduction}
\label{intro}
The subject of this thesis is the use of regression techniques, specifically support vector machine regression, in the analysis and prediction of survival times. The techniques are applied to a dataset of dogs' veterinary records, with the goal of modeling survival from the details of medical visits.

The thesis begins, after the introduction, with the establishment of regression problems and an overview of support vector machines for regression in \autoref{chsvm}. CONTINUES

\chapter{Support Vector Machines for Regression}
\label{chsvm}


\chapter{The Veterinary Dataset}
\label{chdataset}
The real world application of Support Vector Regression explored in this thesis consists in a work of survival analysis on a veterinary dataset.

\section{Dataset specifics}
The dataset, provided as a \texttt{.xlsx} file without formulas, is composed of a collection of dogs' veterinary records, separated in 3 sheets: \textit{2006-2016} data $ (161 \times 30) $, \textit{2001-2005} data $ (69 \times 30) $, and a \textit{legend}. Both data sheets have 30 columns, corresponding to the features enumerated in the legend, but only the one from the \textit{2006-2016} period actually uses all of them. Therefore, only the first sheet has been deemed fit for analysis, hereafter referred to as the \texttt{dogs\_2006\_2016} dataset, with a shape of $ (161\; data\; points \times 30\; features) $.

\section{Feature legend}
Here is the list of all features:
\begin{itemize}
\item \textit{Folder}: patient folder ID;
\item \textit{IP}: presence of Pulmonary Insufficiency (0 = no, 1 = yes);
\item \textit{IP Gravity}: gravity of Pulmonary Insufficiency (0 = absent, 1 = minor, 2 = moderate, 3 = severe);
\item \textit{Vrig Tric}: tricuspid regurgitation velocity;
\item \textit{Birth date}: birth date of patient;
\item \textit{First visit}: date of first medical visit;
\item \textit{Age}: age of patient;
\item \textit{Therapy started}: date of therapy beginning;
\item \textit{Dead}: death of patient during treatment (0 = no, 1 = yes);
\item \textit{Date of death}: date of death, right-censored for alive patients;
\item \textit{MC}: death due to cardiac arrest (0 = no, 1 = yes);
\item \textit{Survival time}: survival time from first visit, right-censored for alive patients;
\item \textit{Furosemide}: prescription of furosemide (0 = no, 1 = yes);
\item \textit{Ache-i}: prescription of acetylcholinesterase inhibitors (0 = no, 1 = yes);
\item \textit{Pimobendan}: prescription of pimobendan (0 = no, 1 = yes);
\item \textit{Spironolattone}: prescription of spironolactone (0 = no, 1 = yes);
\item \textit{Therapy Category}: category of prescriptions, obtained as a sum of the values of features \textit{Furosemide}, \textit{Ache-i}, \textit{Pimobendan} and \textit{Spironolattone};
\item \textit{Antiaritmico}: prescription of antiarrhythmic (0 = no, 1 = yes);
\item \textit{isachc}: ISACHC classification of patient;
\item \textit{Class}: ACVIM classification of patient;
\item \textit{Weight (Kg)}: weight of patient in kilograms;
\item \textit{Asx/Ao}: left atrium / aortic root ratio;
\item \textit{E}: E-wave;
\item \textit{E/A}:	E-wave / A-wave ratio;
\item \textit{FE \%}: ejection fraction;
\item \textit{FS \%}: shortening fraction;
\item \textit{EDVI}: End-diastolic volume of left ventricle indexed for body surface area;
\item \textit{ESVI}: End-systolic volume of left ventricle indexed for body surface area;
\item \textit{Allo diast}: End-diastolic volume of left ventricle not indexed for body surface area;
\item \textit{Allo sist}: End-systolic volume of left ventricle not indexed for body surface area.
\end{itemize}

\chapter{Dataset Preprocessing and Exploration}
\label{chprepr}
Before any meaningful analysis can be carried out, it's important to apply a number of checks and alterations to the raw dataset, so that it becomes suitable for processing. This kind of work, called \textit{preprocessing}, is paramount to successful analysis of real world data, such as the veterinary dataset subject of this thesis.

\section{Consistency check}
The \texttt{dogs\_2006\_2016} dataset was provided as part of a \texttt{.xlsx} sheet without any in-built formulas, so it was deemed sensible to try a number of consistency checks on dependent features, such as dates and times.
\subsection*{Date consistency}
The first check involved testing the validity of the inequality $ Birth\; date \leq First\; visit \leq Therapy\; started \leq Date\; of\; death $. This check produced 42 errors, all from data points for which $ Therapy\; started < First\; visit $, while in all the remaining cases the two values were equal. Then it became clear that the correct interpretation of the data corresponds in fact to the inequality $ Birth\; date \leq Therapy\; started \leq First\; visit \leq Date\; of\; death $. \textit{Therapy started} is equal to \textit{First visit} when the therapy started on the first visit, while it is set to an earlier date if the dog was already undergoing therapy under different supervision.
\subsection*{Survival time consistency}
A check of the \textit{Survival time} as calculated from \textit{First visit} to \textit{Date of death} revealed only one erroneous data point, with an error delta of exactly one year. Since the value was supposedly obtained automatically, the origin of this error is unclear. A decision was made to fix this error in all future analyses by using by default the value obtained from the dates.
\subsection*{Cardiac arrest and death}
A check was performed on whether or not the dataset is consistent with regard to the implication $ CardiacDeath \rightarrow Dead $, although for the purpose of this work any information related to the dog's death such as a cardiac arrest must not be included in the actual prediction models. The check was anyways passed with no reported inconsistencies.
\subsection*{Therapy category}
The \textit{Therapy category} feature was supposedly calculated as the sum of various binary features, each representing a pharmaceutical prescription. A check was made to ensure its correctness and no errors were reported.
\subsection*{Age}
A check on the \textit{Age} value was meant to determine whether it was recorded on the first visit, or on another occasion. Instead, the findings suggest that the value has no consistent correlation with any date feature. Calculating the value from the first visit date yields an error on 50 data points, while changing it to the death date only fixes 5 of these errors. In the end, it was deemed sensible to simply use the newly calculated values obtained with the first visit date, instead of the reported ones.

\section{Exploratory analysis}
After the first checks necessary to assess data consistency, the dataset was further inspected using some exploratory analysis tools. The insights obtained with this techniques proved useful for subsequent tasks such as feature selection and normalization.
\subsection*{Feature correlation}
A correlation heatmap of the available features was used to highlight possible dependencies among them. In particular, Fig. XX shows that the following groups of features exhibit a relevant correlation:
\begin{itemize}
\item \textit{IP Gravity} and \textit{Vrig tric}: a subsequent test confirmed that the first feature is indeed a discretization of the second one;
\item \textit{Dead} and \textit{MC} (Cardiac arrest death);
\item \textit{FE \%} and \textit{FS \%};
\item \textit{EDVI}, \textit{ESVI}, \textit{Allo diast} and \textit{Allo sist} (especially \textit{EDVI} with \textit{Allo diast} and \textit{ESVI} with \textit{Allo sist}).
\end{itemize}
\subsection*{Scatter plots with \textit{Survival time} and other features}
\textit{Survival time} was plotted against all the other features. As shown in Fig. YY, no obvious patterns were found, except for a slight correlation between \textit{Age} and \textit{Asx/ao}.

\section{Handling missing data}
The considered dataset, as is the case with most real world data, contains NA (Not Available) values that need to be handled in some way before being used as input for model training. The following policies were experimented with in order to deal with this problem:
\begin{itemize}
\item \textbf{Dropping NA values}. This procedure consists in simply deleting all data points that include NA values in any feature field. It does not introduce potentially spurious information, but effectively reduces the dataset's size.
\item \textbf{Using the mean value}. This procedure consists in filling any NA value with the mean value of all data points for the corresponding feature. It maintains the dataset's size but it might alter the distributions of features, especially when the number of NA values is high.
\item \textbf{Sampling from a normal distribution}. This technique assumes a normal distribution for the involved features, and replaces NAs with values drawn from an approximation of such distributions. It maintains the dataset's size and feature distributions, but introduces an element of randomness in the analysis. Before applying this technique it's important to verify the normality of the affected features using a tool such as a QQ-plot.
\item \textbf{Using values predicted from a model (e.g., Linear Regression)}. TODO
\end{itemize}

\section{Handling censored data}
A crucial problem in survival analysis is that of data censoring: in this case the \textit{Survival time} feature is right-censored for observations for which the subject is registered as alive. This means that the reported survival time is just a lower bound, while the real time is unknown, either because the subject died after the last visit or because it's still alive.

This problem, discussed more in depth later in this thesis, inspired a modification of the standard Support Vector Machine model. However, some elementary techniques can be applied even at this stage of analysis. Precisely, the following methodologies have been considered:
\begin{itemize}
\item \textbf{Drop censored values}. simply delete all data points for dogs that have not yet died. This technique removes the need to handle censoring in the analysis, but deletes important information at the same time.
\item \textbf{Using the maximum non-censored value}. find the maximum \textit{Survival time} value for dogs that did die and substitute it to all censored values. This technique represents a rudimentary way of letting alive subjects positively influence survival time prediction, as their value is artificially set at the maximum in the resulting dataset.
\end{itemize}

\chapter{Model Training and Selection}
\label{chmodsel}

\section{Cross-validation and repeated holdout}

\section{Feature engineering}


\chapter{Alternative SVM Models for Censored Datasets}
\label{chaltsvm}

\chapter{A Custom SVM Implementation}
\label{chcustsvm}

\chapter{Final Results and Conclusions}
\label{chconcl}

%
%			BIBLIOGRAPHY
%
\begin{thebibliography}{00}
%
\bibitem{itemexample}
Example citation.

%
\end{thebibliography}
% 
\end{document}


 
